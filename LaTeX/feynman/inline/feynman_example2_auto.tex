% !TEX program = pdflatexmk
% !TEX parameter = -shell-escape
\documentclass[a4paper,10pt]{article}
\usepackage[margin=2.4cm]{geometry} % margins
\usepackage{amsmath}
\usepackage{graphicx}

% https://www.overleaf.com/learn/latex/Feynman_diagrams
\usepackage[force]{feynmp-auto}
\newcommand{\ww}{180}
\newcommand{\hh}{150}

\begin{document}

This is a test file for Feynman diagrams.

\begin{figure}[t!]
  \centering
  \begin{fmffile}{compton}
  \begin{fmfgraph*}(150,100)
    \fmfleft{i1,i2}
    \fmfright{o1,o2}
    \fmflabel{$\gamma$}{i2}
    \fmflabel{$e^-$}{i1}
    \fmflabel{$\gamma$}{o1}
    \fmflabel{$e^-$}{o2}
    \fmf{photon}{i2,v2}
    \fmf{fermion}{i1,v1,v2,o2}
    \fmf{photon}{v1,o1}
  \end{fmfgraph*}
  \end{fmffile}
  \vspace{5mm}
  \caption{Feynman diagram for Compton scattering} %\label{compton}
  \vspace{5mm}
\end{figure}

\begin{figure}[t!]
  \centering
  \begin{fmffile}{scat}
  \begin{fmfgraph*}(150,100)
    % Note that the size is given in normal parentheses
    % instead of curly brackets.
    % Define external vertices from bottom to top
    \fmfleft{i1,i2}
    \fmfright{o1,o2}
    \fmflabel{i1}{i1}
    \fmflabel{i2}{i2}
    \fmflabel{o1}{o1}
    \fmflabel{o2}{o2}
    \fmflabel{v1}{v1}
    \fmflabel{v2}{v2}
    \fmf{fermion}{i1,v1,o1}
    \fmf{fermion}{i2,v2,o2}
    \fmf{photon}{v1,v2}
  \end{fmfgraph*}
  \end{fmffile}
  \vspace{5mm}
  \caption{Feynman diagram for scattering} %\label{scat}
  \vspace{5mm}
\end{figure}



\begin{figure}[t!]
  \centering
  \begin{fmffile}{scat2}
  \begin{fmfgraph}(150,150)
    % Note that the size is given in normal parentheses
    % instead of curly brackets.
    % Define external vertices from bottom to top
    \fmfleft{i1,i2}
    \fmfright{o1,o2}
    \fmf{fermion}{i1,v1,o1}
    \fmf{fermion}{i2,v2,o2}
    \fmf{photon}{v1,v2}
  \end{fmfgraph}
  \end{fmffile}
  \vspace{5mm}
  \caption{Second feynman diagram for scattering} %\label{scat2}
  \vspace{5mm}
\end{figure}


%            /o4
%         v22
%        /   \o3
%  i1--v1     
%        \   /o2
%         v21
%            \o1

\begin{figure}[t!]
  \centering
  \begin{fmffile}{higgs1}
  \begin{fmfgraph}(150,150)
    \fmfstraight
    \fmfleft{i1}
    \fmfright{o1,o2,o3,o4}
    \fmf{dashes}{i1,v1}
    \fmf{boson}{v1,v21}
    \fmf{boson}{v1,v22}
    \fmf{fermion}{o1,v21,o2}
    \fmf{fermion}{o3,v22,o4}
  \end{fmfgraph}
  \end{fmffile}
  \vspace{5mm}
  \caption{Feynman diagram for Higgs boson decaying into two W bosons, without phantoms or freezing.}
  \vspace{5mm}
\end{figure}

% Adding \fmf{phantom} makes the bond between the incoming vertices and the interactions tighter and produces a better balanced picture
\begin{figure}[t!]
  \centering
  \begin{fmffile}{higgs2}
  \begin{fmfgraph}(150,150)
    \fmfstraight
    \fmfleft{i0,i1,i2}
    \fmfright{o1,o2,o3,o4}
    \fmf{fermion}{o1,v21,o2}
    \fmf{fermion}{o3,v22,o4}
    \fmf{phantom}{i0,v21}
    \fmf{phantom,tension=0.5}{v21,v22}
    \fmf{phantom}{i2,v22}
    \fmffreeze
    \fmf{dashes,tension=1.5}{i1,v1}
    \fmf{boson}{v1,v21}
    \fmf{boson}{v1,v22}
  \end{fmfgraph}
  \end{fmffile}
  \vspace{5mm}
  \caption{Feynman diagram for Higgs decay with phantoms and freezing to form a skeleton.}
  \vspace{5mm}
\end{figure}

% Adding \fmf{phantom} makes the bond between the incoming vertices and the interactions tighter and produces a better balanced picture
\begin{figure}[t!]
  \centering
  \begin{fmffile}{higgs3}
  \begin{fmfgraph}(150,150)
    \fmfstraight
    \fmfleft{i0,i1,i2}
    \fmfright{o1,o2,o3,o4}
    \fmf{fermion}{o1,v21,o2}
    \fmf{fermion}{o3,v22,o4}
    \fmf{dashes}{i0,v21}
    \fmf{dashes,tension=0.5}{v21,v22}
    \fmf{dashes}{i2,v22}
    \fmffreeze
    \fmf{dashes,tension=1.5}{i1,v1}
    \fmf{boson}{v1,v21}
    \fmf{boson}{v1,v22}
  \end{fmfgraph}
  \end{fmffile}
  \vspace{5mm}
  \caption{Feynman diagram for Higgs decay into a bottom quark pair with dashes for pulling and freezing to form a skeleton.}
  \vspace{5mm}
\end{figure}

\end{document}