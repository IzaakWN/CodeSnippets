\documentclass[10pt,a4paper]{article}
\usepackage[margin=2.4cm]{geometry} % margins
\usepackage{amsmath}
\usepackage{graphicx}
\usepackage{feynmp}

% macro to compile Feynman graphs without extra scripts
\DeclareGraphicsRule{*}{mps}{*}{}
\makeatletter
\def\endfmffile{
  \fmfcmd{\p@rcent\space the end.^^J end.^^J endinput;}
  \if@fmfio
    \immediate\closeout\@outfmf
  \fi
  %\ifnum\pdfshellescape=\@ne
  \ifnum\pdfshellescape>\z@
    \immediate\write18{mpost \thefmffile}
  \fi}
\makeatother



\begin{document}

This is a test file for Feynman diagrams. You need to compile twice: once to make compile and save the Feynman diagrams, twice to include them in the typeset PDF file.

\begin{figure}[h]
  \vspace{10mm}
  \centering
  \begin{fmffile}{compton}
  \begin{fmfgraph*}(150,100)
    \fmfleft{i1,i2}
	\fmfright{o1,o2}
    
	\fmflabel{$\gamma$}{i2}
	\fmflabel{$e^-$}{i1}
	\fmflabel{$\gamma$}{o1}
	\fmflabel{$e^-$}{o2}

	\fmf{photon}{i2,v2}
	\fmf{fermion}{i1,v1,v2,o2}
	\fmf{photon}{v1,o1}
  \end{fmfgraph*}
  \end{fmffile}
  \vspace{5mm}
  \caption{Feynman diagram for Compton scattering} %\label{compton}
\end{figure}

\end{document}