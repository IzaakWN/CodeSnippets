\documentclass[14pt,a4paper]{article}
\usepackage[margin=2.4cm]{geometry} % margins
\usepackage{amsmath}
%\usepackage{bm} % for bold math
\usepackage{graphicx}
\usepackage{subcaption} % for subfigs
\usepackage{hyperref} % hyperlinks [colorlinks=true]
\usepackage{cancel} % MET: $\cancel{\it{E}}_{T}$

\newcommand{\ww}{180}
\newcommand{\hh}{150}

\usepackage{feynmp}
\DeclareGraphicsRule{*}{mps}{*}{}
\makeatletter
\def\endfmffile{%
  \fmfcmd{\p@rcent\space the end.^^J%
          end.^^J%
          endinput;}%
  \if@fmfio
    \immediate\closeout\@outfmf
  \fi
%  \ifnum\pdfshellescape=\@ne
	\ifnum\pdfshellescape>\z@
    \immediate\write18{mpost \thefmffile}%
  \fi}
\makeatother



\begin{document}
%\maketitle

This is a test file for Feynman diagrams.



%  i2??v12???????o2
%       S         
%       S         _t2
%       S        / 
%       S      _t 
%       S     /  \_t1
%       S    /
%  i1??v12??v21??o1

\begin{figure}[t!]
	\LARGE
	\centering
	\begin{fmffile}{ditau}
	\begin{fmfgraph*}(300,200)
    	\fmfset{wiggly_len}{8mm}
		\fmfleft{i1,i2}
		\fmfright{o1,t1,t2,o2}
		\fmf{phantom,tension=1.2,label=b}{i1,v11}
		\fmf{phantom,tension=1.2}{i2,v12}
		\fmf{phantom,tension=1}{i1,v11,o1}
		\fmf{phantom,tension=1}{i2,v12,o2}
		\fmf{boson,tension=1,label=W'}{v11,v12}
		\fmffreeze
		\fmf{fermion}{i1,v11,v21,o1}
		\fmf{fermion,label=$q$}{i2,v12}
		\fmf{fermion,label=$\overline{q}'$,label.side=left}{v12,o2}
		\fmf{phantom,tension=2,label=B'}{v11,v21}
		\fmf{phantom,tension=1.5,label=b}{v21,o1}
		\fmf{boson,label=X(28)}{v21,t}
		\fmf{fermion,label=$\tau^+$}{t1,t}
		\fmf{fermion,label=$\tau^-$}{t,t2}
		\fmf{phantom,tension=1}{v12,t}
		\fmf{phantom,tension=0.4}{v12,v21}
	\end{fmfgraph*}
	\end{fmffile}
	\vspace{5mm}
	\caption{Feynman diagram} %\label{scat}
	\vspace{5mm}
\end{figure}



\end{document}