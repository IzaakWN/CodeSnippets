% !TEX program = pdflatexmk
% !TEX parameter = -shell-escape
% Author: Izaak Neutelings (September 2024)
% Description: Anomalous magnetic moment in pp collisions
% Sources: https://cms.cern.ch/iCMS/analysisadmin/cadilines?line=EXO-23-005
% Instructions: To compile via command line, run the following twice
%   pdflatex -shell-escape anomalous_momentum_arrows.tex
% Sources:
%  https://feynm.net/g-2/#g-2-016
%  https://wiki.physik.uzh.ch/cms/latex:feynman#adding_momentum_arrows
\documentclass[11pt,border=4pt,multi=page,crop]{standalone}
\usepackage{feynmp-auto}
\usepackage{pgffor} % for \foreach


% DEFINE fmfpicture ENVIRONMENT
% The following defines a custom picture environment that
% helps to create standalone pages with common settings,
% and correctly padding the diagram with \fmfframe
\usepackage{environ} % for \NewEnviron
\NewEnviron{fmfpicture}[3]{%
  \begin{page} % to create standalone page
  \fmfframe(#1)(#2){ % padding (LT)(RB)
  \begin{fmffile}{feynmp-#3} % auxiliary files (use unique name!)
    \fmfset{wiggly_len}{12} % boson wavelength
    \fmfset{wiggly_slope}{65} % boson slope of waves
    \BODY % main code
  \end{fmffile}
  }
  \end{page}
}

% MOMENTUM ARROW
\newcommand{\fmfarrow}[5]{%
  %  Arguments:
  %  1. unique label of the arrow object (containing only letters);
  %  2. postion of the momentum arrow w.r.t. the particle leg: `down`, `up`;
  %  3. position of the label w.r.t. momentum arrow (optional): `bot`, `top`;
  %  4. label text drawn with the arrow (optional);
  %  5. comma-separated vertices between which the momentum arrow is drawn.
  \fmfcmd{
    style_def marrow#1
      expr p = drawarrow subpath (0.15, 0.6) of p shifted 6 #2 withpen pencircle scaled 0.4;
      label.#3(btex #4 etex, point 0.3 of p shifted 6 #2);
    enddef;
  }
  \fmf{marrow#1,tension=0}{#5}
}
\newcommand{\fmfbentarrow}[5]{%
  %  Arguments:
  %  1. unique label of the arrow object (containing only letters);
  %  2. postion of the momentum arrow w.r.t. the particle leg: `down`, `up`;
  %  3. position of the label w.r.t. momentum arrow (optional): `bot`, `top`, `lft`;
  %  4. label text drawn with the arrow (optional);
  %  5. comma-separated vertices between which the momentum arrow is drawn.
  \fmfcmd{
    style_def mbentarrow#1
      expr p = drawarrow subpath (0.5, 1.3) of p shifted 8 #2 withpen pencircle scaled 0.4;
      label.#3(btex #4 etex, point 0.8 of p shifted 10 #2);
    enddef;
  }
  \fmf{mbentarrow#1,left=0.6,tension=0}{#5}
}

\begin{document}


% gamma -> ee
\begin{fmfpicture}{15,6}{2,6}{loop} % padding (LT)(RB)
  \begin{fmfgraph*}(80,100) % canvas (W,H)
    % external vertices
    \fmfright{g}
    \fmfleft{f2,f1}
    % main process
    \fmf{boson,tension=0.9}{g,v} % photon
    \fmf{fermion,tension=1.1,label.distance=2,
         label.side=right,label=$k$}{v2,v} % internal lepton
    \fmf{fermion,tension=1.1,label.distance=1,
         label.side=right,label=$k'=k+p$}{v,v1} % internal lepton
    \fmf{fermion,label.distance=2,
         label.side=right,label=$p$}{v1,f1} % incoming lepton
    \fmf{fermion,label.distance=2,
         label.side=right,label=$p'$}{f2,v2} % outgoing lepton
    % virtual photon
    \fmffreeze % freeze to fix fermion lines
    \fmf{boson,right=0.6}{v1,v2} %,label=$\gamma$,label.side=right
    % ARROWS
    \fmfarrow{q}{down}{bot}{$q$}{g,v}
    \fmfbentarrow{k}{left}{lft}{$p-k$}{v2,v1}
    % labels
    %\fmfv{l.a=-120,l.d=8,l=$\gamma$}{g}
    %\fmfv{l.a=140,l.d=4,l=e}{f1}
    %\fmfv{l.a=-155,l.d=4,l=e}{f2}
  \end{fmfgraph*}
\end{fmfpicture}


\end{document}