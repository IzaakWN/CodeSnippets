% !TEX program = pdflatexmk
% !TEX parameter = -shell-escape
% Author: Izaak Neutelings (May 2023)
% Instructions: To compile via command line, run the following twice
%   pdflatex -shell-escape single_top.tex
\documentclass[10pt,border=2pt,multi=page,crop]{standalone}
\usepackage{amsmath}
\usepackage{feynmp-auto}
\usepackage{xcolor}

% DEFINE TEXT COLORS
%\definecolor{collep}{rgb}{.1,.6,.1} % lepton (green)
\definecolor{collep}{rgb}{.74,.05,.05} % lepton (red)
\definecolor{colrad}{rgb}{.20,.41,.90} % radiation (cyan)
%\definecolor{colvtx}{rgb}{.1,.1,.7} % vertex (dark blue)
%\definecolor{mybl}{rgb}{0,0,1} % blue
%\definecolor{mygr}{rgb}{.1,.7,.1} % green
%\definecolor{myor}{rgb}{1,0.5,0} % orange

% DEFINE COLOR MACROS
% The following loops over the user color names and defines
% a handy \<colname> command to set text color, as well as
% defines colors in MetaPost of the same and value for lines
\usepackage{pgffor} % for \foreach
\def\MPcolors{} % MetaPost code importing xcolor names
\foreach \colname in {colkappaf,colkappaV,colkappah}{ % create command & MetaPost code
  %\expandafter\xdef\csname\colname\endcsname{\noexpand\color{\colname}}% \newcommand\<colname>
  \convertcolorspec{named}{\colname}{rgb}\tmprgb % get rgb code
  \xdef\MPcolors{\MPcolors color \colname; \colname := (\tmprgb); } % add color name
}

% DEFINE fmfpicture ENVIRONMENT
% The following defines a custom picture environment that
% helps to create standalone pages with common settings,
% and correctly padding the diagram with \fmfframe
\usepackage{environ} % for \NewEnviron
\NewEnviron{fmfpicture}[3]{%
  \begin{page} % to create standalone page
  \fmfframe(#1)(#2){ % padding (LT)(RB)
  \begin{fmffile}{feynmp-#3} % auxiliary files (use unique name!)
    \fmfset{wiggly_len}{12} % boson wavelength
    \fmfset{wiggly_slope}{65} % boson slope of waves
    \fmfcmd\MPcolors % define custom line colors in MetaPost (does not work in \fmfv)
    \BODY % main code
  \end{fmffile}
  }
  \end{page}
}

% LOOP MACRO
\def\foreachlep#1{\foreach \lep in {\mathrm{e},\mu,\tau,\ell}{#1}}
%\def\foreachlep#1{\foreach \lep in {\ell,\tau}{#1}}

\begin{document}
\large

% DRELL-YAN - Quark-antiquark annihilation
\foreachlep{
\begin{fmfpicture}{-8,12}{-1,8}{feynmp-dy-qq} % padding (LTRB)
  \begin{fmfgraph*}(110,60)
    % external vertices
    \fmfleft{i2,i1}
    \fmfright{o2,o1}
    % main
    \fmf{fermion}{i1,v1,i2}
    \fmf{boson,l.s=right,l.d=7,label=Z/$\gamma^*$}{v2,v1} % s-channel
    \fmf{fermion,f=collep}{o2,v2,o1}
    % labels
    \fmfv{l.a=160,l.d=4,l=$q$}{i1}
    \fmfv{l.a=-158,l.d=4,l=$\overline{q}$}{i2}
    \fmfv{l.a=24,l.d=4,l=\collep$\lep^-$}{o1}
    \fmfv{l.a=-22,l.d=4,l=\collep$\lep^+$}{o2}
  \end{fmfgraph*}
\end{fmfpicture}}


% DRELL-YAN - gluon radiation (no decay)
\begin{fmfpicture}{-8,10}{-4,10}{feynmp-dy-qq-g} % padding (LTRB)
  \begin{fmfgraph*}(110,60)
    % external vertices
    \fmfleft{i2,i1} 
    \fmfright{o2,o1}
    % main
    \fmf{fermion,t=1.0}{i1,v1}
    \fmf{fermion,t=0.8}{v1,v2} % t-channel
    \fmf{fermion,t=1.0}{v2,i2}
    \fmf{photon}{v1,o1}
    \fmf{gluon}{v2,o2}
    % labels
    \fmfv{l.a=160,l=$q$}{i1}
    \fmfv{l.a=-158,l=$\overline{q}$}{i2}
    \fmfv{l.a=22,l=Z}{o1}
    \fmfv{l.a=-24,l=$g$}{o2}
  \end{fmfgraph*}
\end{fmfpicture}


% DRELL-YAN - quark radiation
\foreachlep{
\begin{fmfpicture}{-8,10}{-4,10}{feynmp-dy-qq-mumug} % padding (LTRB)
  \begin{fmfgraph*}(125,75)
    \fmfleft{i2,i1}
    \fmfright{o2,o1}
    % skeleton
    \fmf{phantom}{i1,v1,i2}
    \fmf{phantom}{o1,v2,o2}
    \fmf{phantom}{v1,v2}
    \fmffreeze
    \fmftop{t}
    \fmfshift{5 left}{i1}
    \fmfshift{5 up}{i1,t}
    % main
    \fmf{gluon,tension=0.8}{g,i1}
    \fmf{fermion}{v1,g}
    \fmf{fermion}{i2,v1}
    \fmf{fermion,f=colrad,tension=0}{g,t}
    \fmf{boson,label=Z/$\gamma^*$,label.side=right}{v1,v2}
    \fmf{fermion,f=collep}{o2,v2,o1}
    % labels
    \fmfv{l.a=160,l.d=4,l=$g$}{i1}
    \fmfv{l.a=-154,l.d=4,l=$q$}{i2}
    \fmfv{l.a=24,l.d=4,l=\collep$\lep^-$}{o1}
    \fmfv{l.a=-22,l.d=4,l=\collep$\lep^+$}{o2}
    %\fmfv{l.a=10,l.d=5,l=\colrad$q$}{t}
    \fmfv{l.a=-110,l.d=10,l=\colrad$q$}{t}
  \end{fmfgraph*}
\end{fmfpicture}
}


% DRELL-YAN - qq -> qqZ -> qqtautau
\begin{fmfpicture}{-7,10}{10,10}{qq-qqZ-qqtautau} % padding (LTRB)
  \begin{fmfgraph*}(135,80) % dimensions (WH)
    % external vertices
    \fmfleft{i2,i1}
    \fmfright{o2,t2,t1,o1}
    % skeleton
    \fmf{fermion}{i1,v1,o1}
    \fmf{fermion}{i2,v2,o2}
    \fmf{phantom,t=0.25}{v1,v2} % pull Vqq vertices together
    \fmf{phantom,t=0.60}{v1,i1,i2,v2} % pull Vqq vertices to left
    \fmffreeze
    % VV -> H -> HH
    \fmf{boson,t=0.9}{v2,v,v1}
    \fmf{boson,t=1.1,l.s=left,label=Z}{v,z} % Z boson
    \fmf{fermion}{t2,z,t1}
    \fmffreeze
    % labels
    \fmf{phantom,t=6}{w1,v,w2} % create extra vertices for labels
    \fmf{phantom,l.d=4,l.s=right,label=$W$}{v1,w1}
    \fmf{phantom,l.d=2,l.s=left,label=$W$}{v2,w2}
    \fmfv{l.a=158,l.d=6,l=\strut$q$}{i1}
    \fmfv{l.a=-158,l.d=6,l=\strut$q'$}{i2}
    \fmfv{l.a=22,l.d=6,l=\strut$q$}{o1}
    \fmfv{l.a=-22,l.d=6,l=\strut$q'$}{o2}
    \fmfv{l.a=25,l.d=4,l=$\tau^-$}{t1}
    \fmfv{l.a=-20,l.d=4,l=$\tau^+$}{t2}
  \end{fmfgraph*}
\end{fmfpicture}


\end{document}