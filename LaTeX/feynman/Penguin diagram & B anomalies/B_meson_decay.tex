% !TEX program = pdflatexmk
% !TEX parameter = -shell-escape
% Author: Izaak Neutelings (September 2024)
% Instructions: To compile via command line, run the following twice
%   pdflatex -shell-escape pp_hard.tex
\documentclass[11pt,border=2pt,multi=page,crop]{standalone}
\usepackage{graphicx}
\usepackage{feynmp-auto}

% DEFINE TEXT COLORS
\usepackage{xcolor}
\definecolor{colvtx}{rgb}{0,.1,1} % LQ vertex (blue)
\definecolor{colkappaf}{rgb}{1,0,0} % kappa_f (red)

% DEFINE COLOR MACROS
% The following loops over the user color names and defines
% a handy \<colname> command to set text color, as well as
% defines colors in MetaPost of the same and value for lines
\usepackage{pgffor} % for \foreach
\def\MPcolors{} % MetaPost code importing xcolor names
\foreach \colname in {colvtx,colkappaf}{ % create command & MetaPost code
  \expandafter\xdef\csname\colname\endcsname{\noexpand\color{\colname}} % \newcommand\<colname>
  \convertcolorspec{named}{\colname}{rgb}\tmprgb % get rgb code
  \xdef\MPcolors{\MPcolors color \colname; \colname := (\tmprgb); } % add color name
}

% DEFINE fmfpicture ENVIRONMENT
\usepackage{environ} % for \NewEnviron
\NewEnviron{fmfpicture}[3]{%
  \begin{page} % to create standalone page
  \fmfframe(#1)(#2){ % padding (LT)(RB)
  \begin{fmffile}{feynmp-#3} % auxiliary files (use unp2ue name!)
    \fmfset{arrow_len}{10} % arrow length
    \fmfset{dash_len}{8} % dashes length
    \fmfset{wiggly_len}{11} % boson wavelength
    \fmfset{wiggly_slope}{65} % boson slope of waves
    \fmfcmd\MPcolors % define custom line colors in MetaPost
    \BODY % main code
  \end{fmffile}
  }
  \end{page}
}

% CUSTOM DRAW MACROS
% https://wiki.physik.uzh.ch/cms/latex:feynman
% https://ctan.math.illinois.edu/macros/latex/contrib/feynmf/fmfman.pdf
\usepackage{listofitems}
\def\quarknew#1#2{
  \readlist*\mylist{#1}
  \readlist*\mylisttwo{#2}
  %\fmfpair{#1,#2}
  \fmfforce{vloc(__\mylisttwo[1])+(0,-10)}{\mylist[1]}
  \fmfcmd{
    %pair a1, b, m, n;
    pair a, b;
    a = (0,0);
    b = a + (20,20);
    draw vloc(__i3) -- b;
  %  %idraw ("",
  %    path q; q = (0,0) -- (80,50);
  %    draw q;
%  a = point 0 of p;
%  b = point length(p) of p + (#7);
%  m = point length(p) of p + (#8);
%  cdraw subpath (#3) of q shifted (#5);
%  % arrow heads
%  frac := (xpart(#3)+ypart(#3))*0.46*xpart(#4); % fraction along path
%  if substring (0,1) of "#6" = "-":
%    cfill (tarrow (reverse(q),frac)) shifted (#5);
%  else:
%    cfill (tarrow (q,frac)) shifted (#5);
%  fi;
    }
}
\newcommand{\quark}[9]{
  \fmfcmd{ %input TEX; input latexmp; setupLaTeXMP(packages="amssymb,amsmath");
    style_def quark#1 expr p =
    pair a, b, m, n;
    if (substring (0,4) of "#6" = "left") or (substring (1,5) of "#6" = "left"):
      a = point 0 of p; b = point length(p) of p + (#7); m = point length(p) of p + (#8);
      path q; q = a{m-a}..tension ypart(#4)..{right}b;
      label.lft(btex #2 etex, point xpart(#3) of q shifted (#5))
    fi;
    if (substring (0,5) of "#6" = "right") or (substring (1,6) of "#6" = "right"):
      a = point 0 of p + (#7); b = point length(p) of p; m = point 0 of p + (#8);
      path q; q = a{right}..tension ypart(#4)..{b-m}b;
      label.rt(btex #2 etex, point ypart(#3) of q shifted (#5))
    fi;
    cdraw subpath (#3) of q shifted (#5);
    % arrow heads
    frac := (xpart(#3)+ypart(#3))*0.46*xpart(#4); % fraction along path
    if substring (0,1) of "#6" = "-":
      cfill (tarrow (reverse(q),frac)) shifted (#5);
    else:
      cfill (tarrow (q,frac)) shifted (#5);
    fi;
    enddef;}
  \fmf{quark#1,tension=0}{#9}
}

% BRACE
\usepackage{scalerel}
\newcommand{\mylbrace}[2]{\vspace{#2pt}\hspace{4pt}\scaleleftright[\dimexpr6pt+#1\dimexpr0.11pt]{\lbrace}{\rule[\dimexpr2pt-#1\dimexpr0.5pt]{-4pt}{#1pt}}{.}}
\newcommand{\myrbrace}[2]{\vspace{#2pt}\scaleleftright[\dimexpr6pt+#1\dimexpr0.11pt]{.}{\rule[\dimexpr2pt-#1\dimexpr0.5pt]{-4pt}{#1pt}}{\rbrace}\hspace{4pt}}

\begin{document}


% B MESON DECAY
\begin{fmfpicture}{30,15}{45,25}{pp_hard_DY} % padding (LTRB)
  \begin{fmfgraph*}(120,80) % dimensions (WH)
    % external vertices
    \fmfstraight
    \fmfleft{i3,i1}
    \fmfright{o3,o2,o1}
    % fermions
    \fmflabel{$\ell^\pm$}{o1}
    \fmflabel{$\nu_\ell$}{o2}
    \fmf{phantom}{i1,v1,o1}
    \fmf{fermion,tension=0}{o2,v1,o1}
    % boson
    \fmf{boson,label=$\mathrm{W}^\pm$,label.side=left}{v3,v1}
    % neutron
    \fmf{phantom}{i3,v3,o3} % to help \quark
    \fmffreeze
    % vertices
    \fmfv{l=\parbox{10mm}{\centering B meson}\mylbrace{26}{-3},l.d=16,l.a=-160}{i3}
    \fmfv{l=\myrbrace{26}{-3}\parbox{10mm}{\centering D meson},l.d=16,l.a=-20}{o3}
    %\fmfiv{l=$\overline{\mathrm{b}}$,l.d=3,l.a=180}{vloc(__i3)}
    %\fmfiv{l=$\overline{\mathrm{c}}$,l.d=3,l.a=  0}{vloc(__o3)}
    % spectator quarks
    \fmf{fermion}{i3,v3,o3}
%    \quark{qai}{$b$}{0,1}{1.00,infinity}{0,  0}{left}{0, 0}{0,0}{i3,v3}
%    \quark{qbi}{$q$}{0,1}{1.09,1}       {0,-14}{left}{0,-6}{0,0}{i3,v3}
%    \quark{qao}{$c$}{0,1}{1.00,infinity}{0,  0}{right}{0, 0}{0,0}{v3,o3}
%    \quark{qbo}{$q$}{0,1}{0.84,1}       {0,-14}{right}{0,-6}{0,0}{v3,o3}
%    \quark{qai}{$b$}{0,1}{1.00,infinity}{0,  0}{left}{0, 0}{0,0}{i3,v3}
%    \quark{qbi}{$q$}{0,1}{1.09,1}{0,-14}{left}{0,-6}{0,0}{i3,v3}
    
    %\fmfforce{vloc(__i3)+(0,-10)}{sp}
    \fmfv{d.sh=circle,d.f=1,d.si=3,l=sp,l.a=40}{sp}
    
    \quarknew{sp,sq}{i3,v3}
    %\quarknew{}{sp,sq}{i3,v3,o3}
    
    % labels
  \end{fmfgraph*}
\end{fmfpicture}



\begin{fmfpicture}{30,15}{45,25}{pp_test} % padding (LTRB)
  \begin{fmfgraph*}(140,60) % dimensions (WH)
    \fmfipair{i,o,v,m}
    \fmfiequ{i}{(0,0)}
    \fmfiequ{o}{(w,0)}
    \fmfiequ{v}{(.5w,.5h)}
    \fmfiequ{m}{(.5w,h)}
    % grey tangent line (for illustration purposes)
    \fmfi{dashes,foreground=(0.7,,0.7,,0.7)}{m--i}
    \fmfi{dashes,foreground=(0.7,,0.7,,0.7)}{o--m}
    % curved paths
    \fmfi{fermion}{i{(m+(0,-10))-i} .. {right}v}
    \fmfi{fermion}{v{right} .. {o-m}o}
    % point labels
    \fmfiv{d.sh=circle,d.f=1,d.si=2pt,l=i}{i}
    \fmfiv{d.sh=circle,d.f=1,d.si=2pt,l=o}{o}
    \fmfiv{d.sh=circle,d.f=1,d.si=2pt,l=m}{m}
    \fmfiv{d.sh=circle,d.f=1,d.si=2pt,l=v,l.a=90}{v}
  \end{fmfgraph*}
\end{fmfpicture}


\end{document}