% Behaviour of the stator voltage and the RMS-value of the stator
% flux as a function of speed in scalar control.
% Author: Erno Pentzin (2013)
\documentclass{article}
\usepackage{amsmath} % for \text
\usepackage{tikz}
\tikzset{>=latex} % for LaTeX arrow head
\definecolor{mylightred}{RGB}{255,200,200}
\definecolor{mylightblue}{RGB}{172,188,63}
\definecolor{mylightgreen}{RGB}{150,220,150}

% split figures into pages
\usepackage[active,tightpage]{preview}
\PreviewEnvironment{tikzpicture}
\setlength\PreviewBorder{1pt}%

\begin{document}



\begin{tikzpicture}[scale=6]

  \def\isoe{0.15}
  \def\isomu{0.20}
  \def\isoSB{0.50}
  \def\isomax{0.60}
  
  % axes
  \draw[->,thick]
    (0,0) -- (0,\isomax)
    node[left=32pt,anchor=north east,rotate=90] {muon isolation $I_\mu$};
  \draw[->,thick]
    (0,0) -- (\isomax,0)
    node[below=7pt,anchor=north east] {electron isolation $I_\text{e}$};
  
  % boxes
  \draw[thick,fill=mylightgreen]
    (0,0) rectangle (\isoSB,\isoSB)
    node[anchor=north east] {SB};
  \draw[thick,fill=mylightred]
    (0,0) rectangle (\isoe,\isomu)
    node[anchor=north east] {SR};
  
  
  % labels
  \draw[fill=blue]
    (0,\isomu)  node[anchor=east]  {\scriptsize$\isomu$}
    (0,\isoSB)  node[anchor=east]  {\scriptsize$\isoSB$}
    (\isoe, 0)  node[anchor=north] {\scriptsize$\isoe$}
    (\isoSB,0)  node[anchor=north] {\scriptsize$\isoSB$};

\end{tikzpicture}



\begin{tikzpicture}

  % horizontal axis
  \draw[->] (0,0) -- (6,0) node[anchor=north] {$f/f_N$};
  % labels
  \draw	(0,0) node[anchor=north] {0}
        (2,0) node[anchor=north] {1}
        (4,0) node[anchor=north] {2};
  % ranges
  \draw (1,3.5) node{{\scriptsize Constant flux}}
        (4,3.5) node{{\scriptsize Field weakening}};
  
  % vertical axis
  \draw[->] (0,0) -- (0,4) node[anchor=east] {$U_s,\Psi_s$};
  %\draw[->] (0,0) -- (0,4) node[anchor=east] {$U_s,\varPsi_s$};
  % nominal speed
  \draw[dotted] (2,0) -- (2,4);
  
  % Us
  \draw[thick] (0,0) -- (2,2) -- (6,2);
  \draw (1,1.5) node {$U_s$}; %label
  
  % Psis
  \draw[thick,dashed] (0,3) -- (2,3) parabola[bend at end] (6,1);
  \draw (2.5,3) node {$\Psi_s$}; %label
  %\draw (2.5,3) node {$\varPsi_s$}; %label
  
\end{tikzpicture}



\begin{tikzpicture}[scale=3]
  
  % limits
  \def\N{4}
  \def\R{1.2}
  
  % axis labels
  \node[scale=0.8,below left=1pt] at (0,\R) {$y$};
  \node[scale=0.8,below left=1pt] at (\R,0) {$z$};
  
  % lines
  \foreach \t/\e in {90/0,60/0.55,45/0.88,30/1.32,10/2.44,0/\infty}{
    \draw[->,black!60!red,thick] %samples=\N,variable=\x,domain=0:1]
      (0,0) -- ({\R*cos(\t)},{\R*sin(\t)})
      %plot({\x*\R*cos(\t)},{\x*\R*sin(\t)}) % alternative way with plotting a linear function
      node[anchor=180+\t,black] {$\eta=\e$};
    \node[fill=white,scale=0.8] at ({0.8*cos(\t)},{0.8*sin(\t)}) {$\theta=\t^\circ$};
  }
    
\end{tikzpicture}



\end{document}