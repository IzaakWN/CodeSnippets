% A cone in 3D
% Author: Marco Daniel

\documentclass[border=3pt,tikz]{standalone}

\usepackage{tikz}
\usepackage{tikz-3dplot}
\tikzset{ % for carc in pic
  pics/carc/.style args={#1:#2:#3}{
    code={
      \draw[pic actions] (#1:#3) arc(#1:#2:#3);
}}}

%%%%<
%\usepackage{verbatim}
%\usepackage[active,tightpage]{preview}
%\PreviewEnvironment{tikzpicture}
%\setlength\PreviewBorder{5pt}%
%%%%>

%\begin{comment}
%:Title: A Cone in 3D
%:Tags: 2D,geometry,mathematics
%:Author: Marco Daniel
%:Slug: cone
%
%\end{comment}

\begin{document}




\tdplotsetmaincoords{70}{0}
\begin{tikzpicture}[tdplot_main_coords]
\def\RI{2}
\def\RII{1.25}

\draw[thick] (\RI,0)
  \foreach \x in {0,300,240,180} { --  (\x:\RI) node at (\x:\RI) (R1-\x) {\x} };

\end{tikzpicture}



\tdplotsetmaincoords{70}{0}
\begin{tikzpicture}[tdplot_main_coords]
\def\RI{2}
\def\RII{1.25}

\draw[dashed,thick] (R1-0.center)
  \foreach \x in {60,120,180} { --  (\x:\RI) node at (\x:\RI) (R1-\x) {\x} };

\end{tikzpicture}



\tdplotsetmaincoords{70}{0}
\begin{tikzpicture}[tdplot_main_coords]
\def\RI{2}
\def\RII{1.25}

\draw[dashed,thick] (R1-0.center)
  \foreach \x in {60,120,180} { --  (\x:\RI) node at (\x:\RI) (R1-\x) {\x} };

%\draw[->] % draw arc beginning from some point
%  (60:2) arc[start angle=-90, end angle=180, radius=0.5] node[left] {$\psi$};
\draw[thick,->] (60:2) pic{carc=180:300:1}; % draw arc from some center point

\end{tikzpicture}


\tdplotsetmaincoords{70}{0}
\begin{tikzpicture}[tdplot_main_coords]
\def\RI{2}
\def\RII{1.25}

\draw[thick] (\RI,0)
  \foreach \x in {0,300,240,180} { --  (\x:\RI) node at (\x:\RI) (R1-\x) {} };
\draw[dashed,thick] (R1-0.center)
  \foreach \x in {60,120,180} { --  (\x:\RI) node at (\x:\RI) (R1-\x) {} };
\path[fill=gray!30] (\RI,0)
  \foreach \x in {0,60,120,180,240,300} { --  (\x:\RI)};

\end{tikzpicture}




\tdplotsetmaincoords{70}{0}
\begin{tikzpicture}[tdplot_main_coords]
\def\RI{2}
\def\RII{1.25}

\begin{scope}[yshift=2cm]
\draw[thick,fill=gray!30,opacity=0.2] (\RII,0)
  \foreach \x in {0,60,120,180,240,300,360} { --  (\x:\RII) node at (\x:\RII) (R2-\x) {}};
\end{scope}

\end{tikzpicture}



\tdplotsetmaincoords{70}{0}
\begin{tikzpicture}[tdplot_main_coords]
\def\RI{2}
\def\RII{1.25}

\draw[thick] (\RI,0)
  \foreach \x in {0,300,240,180} { --  (\x:\RI) node at (\x:\RI) (R1-\x) {} };
\draw[dashed,thick] (R1-0.center)
  \foreach \x in {60,120,180} { --  (\x:\RI) node at (\x:\RI) (R1-\x) {} };
\path[fill=gray!30] (\RI,0)
  \foreach \x in {0,60,120,180,240,300} { --  (\x:\RI)};

\begin{scope}[yshift=2cm]
\draw[thick,fill=gray!30,opacity=0.2] (\RII,0)
  \foreach \x in {0,60,120,180,240,300,360} { --  (\x:\RII) node at (\x:\RII) (R2-\x) {}};
\end{scope}

\foreach \x in {0,180,240,300} { \draw (R1-\x.center)--(R2-\x.center); };
\foreach \x in {60,120} { \draw[dashed] (R1-\x.center)--(R2-\x.center); };
\end{tikzpicture}

\end{document}