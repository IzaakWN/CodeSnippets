% Author: Izaak Neutelings (June 2017)
% taken from https://tex.stackexchange.com/questions/159445/draw-in-cylindrical-and-spherical-coordinates

\documentclass[border=3pt,tikz]{standalone}
\usepackage{physics}
\usepackage{tikz}
\usepackage{tikz-3dplot}
\usepackage[outline]{contour} % glow around text
\usepackage{xcolor}

\colorlet{veccol}{green!50!black}
\colorlet{projcol}{blue!70!black}
\colorlet{myblue}{blue!70!black}
\colorlet{xcol}{blue!85!black}
\tikzset{>=latex} % for LaTeX arrow head
\tikzstyle{proj}=[projcol!80,line width=0.08] %very thin
\tikzstyle{area}=[draw=veccol,fill=veccol!80,fill opacity=0.6]
\tikzstyle{vector}=[->,veccol,thick]
\usetikzlibrary{angles,quotes} % for pic (angle labels)
\contourlength{1.3pt}

\begin{document}



% 3D AXIS with spherical coordinates
\tdplotsetmaincoords{60}{105}
\begin{tikzpicture}[scale=2.0,tdplot_main_coords]
  
  % VARIABLES
  \def\l{0.3}
  \def\rvec{1.2}
  \def\thetavec{46}
  \def\phivec{50}
  
  % AXES
  \coordinate (O) at (0,0,0);
  \coordinate (A) at (0.2,-0.1,0.9);
  \coordinate (P1) at (0.3,0.2,0.85);
  \coordinate (P2) at (0.8,1.0,0.6);
  \coordinate (B) at (0.9,1.2,0.3);
  %\tdplotsetcoord{P1}{\rvec}{\thetavec}{\phivec}
  %\tdplotsetcoord{P2}{\rvec}{\thetavec}{\phivec}
  \draw[thick,->] (0,0,0) -- (1,0,0) node[below left=-2]{$x$};
  \draw[thick,->] (0,0,0) -- (0,1,0) node[right=-1]{$y$};
  \draw[thick,->] (0,0,0) -- (0,0,1) node[above=-1]{$z$};
  %\draw[vector] (0,0,0) -- (1.3*\l,0,0) node[above=3,left=-1,scale=0.8]{$\vu{x}$};
  %\draw[vector] (0,0,0) -- (0,.9*\l,0) node[right=2,above=-1,scale=0.8]{$\vu{y}$};
  %\draw[vector] (0,0,0) -- (0,0,\l) node[left,scale=0.8]{$\vu{z}$};
  
  % PATH
  \draw[xcol!80]
    (A) to[out=-45,in=160] (P1) to[out=-20,in=100] (P2) to[out=-80,in=150] (B);
  
  % VECTORS
  \node[circle,inner sep=0.9,fill=xcol!80!black] (P1') at (P1) {};
  \node[circle,inner sep=0.9,fill=xcol!80!black] (P2') at (P2) {};
  \draw[-stealth,thick,xcol] (O)  -- (P1')
    node[xcol,midway,above=4,right=-1,scale=0.8] {$\vb{r}(t_1)$}
    node[xcol!80!black,above right=-2] {$\mathrm{P}_1$};
  \draw[-stealth,thick,xcol] (O)  -- (P2')
    node[xcol,midway,right=4,above=-2,scale=0.8] {$\vb{r}(t_2)$}
    node[xcol!80!black,above right=-2] {$\mathrm{P}_2$};

\end{tikzpicture}



\end{document}