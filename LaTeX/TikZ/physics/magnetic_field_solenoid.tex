% Author: Izaak Neutelings (March 2020)
\documentclass[border=3pt,tikz]{standalone}
\usepackage{amsmath} % for \dfrac
\usepackage{physics}
\usepackage{ifthen}
\usepackage{tikz,pgfplots}
\usepackage{tikz-3dplot}
\usepackage{auto-pst-pdf}
\usepackage{pst-magneticfield}
\usepackage[outline]{contour} % glow around text
\usetikzlibrary{angles,quotes} % for pic (angle labels)
\usetikzlibrary{arrows,arrows.meta}
\usetikzlibrary{calc}
\usetikzlibrary{decorations.markings}
\tikzset{>=latex} % for LaTeX arrow head
\usepackage{xcolor}
\colorlet{Ecol}{orange!90!black}
\colorlet{EcolFL}{orange!80!black}
\colorlet{veccol}{green!45!black}
\colorlet{projcol}{blue!70!black}
\colorlet{EFcol}{red!60!black}
\colorlet{Bcol}{violet!90}
\colorlet{Bcol1}{violet!80!blue!90}
\colorlet{Bcol2}{violet!80!red!90}
\colorlet{BFcol}{red!70!black}
\colorlet{veccol}{green!45!black}
\colorlet{Icol}{blue!70!black}
\colorlet{Ampcol}{green!60!black!70}
\tikzstyle{BField}=[->,thick,Bcol]
\tikzstyle{current}=[->,Icol,thick]
\tikzstyle{force}=[->,thick,BFcol]
\tikzstyle{vector}=[->,thick,veccol]
\tikzstyle{velocity}=[->,very thick,vcol]
\tikzstyle{metal}=[top color=black!15,bottom color=black!25,middle color=black!20,shading angle=10]
\tikzstyle{darkmetal}=[top color=black!40,bottom color=black!70,middle color=black!30,shading angle=10]
\tikzstyle{lightmetal}=[thin,black!20,top color=black!3,bottom color=black!6,middle color=black!1,shading angle=10]
\tikzstyle{proj}=[projcol!80,line width=0.08] %very thin
\tikzstyle{area}=[draw=veccol,fill=veccol!80,fill opacity=0.6]
\tikzstyle{measline}=[{Latex[length=3]}-{Latex[length=3]}]
\tikzset{
  BFieldLine/.style={thick,Bcol,decoration={markings,mark=at position #1 with {\arrow{latex}}},
                                postaction={decorate}},
  BFieldLine/.default=0.5,
  Ampcurve/.style={Ampcol,decoration={markings,mark=at position #1 with {\arrow{latex}}},
                          postaction={decorate}},
  Ampcurve/.default=0.55,
  pics/Bin/.style={
    code={
      \def\R{0.12}
      \draw[pic actions,line width=0.6,#1,fill=white] % ,thick
        (0,0) circle (\R) (-135:.75*\R) -- (45:.75*\R) (-45:.75*\R) -- (135:.75*\R);
  }},
  pics/Bout/.style={
    code={
      \def\R{0.12}
      \draw[pic actions,line width=0.6,#1,fill=white] (0,0) circle (\R);
      \fill[pic actions,#1] (0,0) circle (0.3*\R);
  }},
  pics/Bin/.default=Bcol,
  pics/Bout/.default=Bcol,
}
\tikzstyle{measure}=[fill=white,midway,outer sep=2]
%\newcommand\arrmark[1]{mark=at position #1 with {\arrow{latex}}}
%\def\myarrmark#1{mark=at position #1 with {\arrow{latex}}}
\contourlength{1.4pt}

% RING SHADING
\makeatletter
\pgfdeclareradialshading[tikz@ball]{ring}{\pgfpoint{0cm}{0cm}}%
{rgb(0cm)=(1,1,1);
rgb(0.719cm)=(1,1,1);
color(0.72cm)=(tikz@ball);
rgb(0.9cm)=(1,1,1)}
\tikzoption{ring color}{\pgfutil@colorlet{tikz@ball}{#1}\def\tikz@shading{ring}\tikz@addmode{\tikz@mode@shadetrue}}
\makeatother


\begin{document}


% SOLENOID MAGNETIC FIELD + Amp�re's law
\contourlength{1.0pt}
\begin{tikzpicture}
  \message{Loop with Ampere's law start. ^^J}
  \def\R{1}
  \def\N{12}
  \def\NB{3}
  \def\L{6}
  \def\t{0.20*\R}
  \def\w{\L/(\N-1)}
  %\draw[current] (-0.6*\R,-0.01*\R) arc (-94:-86:{0.6*\R/sin(4)})
  %  node[midway,below=0.6] {\contour{white}{$I$}};
  
  % WIRE BACK
  \foreach \i [evaluate={\x=-\L/2+(\i-1)*\w; \ang=atan(4*\R/(\w));}] in {1,...,\N}{
    \ifthenelse{\i<\N}{
      \draw[lightmetal]
        ({\x+\w/2},-\R)++(\ang+90:\t/2) to[out=\ang+5,in=\ang-185]++ (\ang:2.02*\R) -- ({\x+\w},\R)
                                        --++ (\ang-90:\t/2) to[out=\ang-185,in=\ang+5]++ (\ang-180:2.02*\R) -- cycle;
    }{}
  }
  
  % MAGNETIC FIELDLINES
  \foreach \i [evaluate={\y=(\i-\NB/2-0.5)*1.4*\R/\NB}] in {1,...,\NB}{
      \draw[BFieldLine={0.505}] (-0.55*\L,\y) -- (0.62*\L,\y);
  }
  \node[Bcol] at (0.61*\L,0.76*\R) {$\vb{B}$};
  
  % AMPERE's LOOP
  \draw[Ampcol]
    (-0.27*\L,0.6*\R) -| (0.27*\L,1.5*\R) -| cycle;
  \draw[Ampcurve={0.54}]
    (-0.27*\L,0.6*\R) -- (0.27*\L,0.6*\R) node[midway,above,scale=0.8] {1};
  \draw[Ampcurve={0.68}]
    ( 0.27*\L,0.6*\R) -- (0.27*\L,1.5*\R) node[midway,above=3,above right=-1,scale=0.8] {2};
  \draw[Ampcurve={0.55}]
    ( 0.27*\L,1.5*\R) -- (-0.27*\L,1.5*\R) node[midway,above=3,above=-1,scale=0.8] {3};
  \draw[Ampcurve={0.57}]
    (-0.27*\L,1.5*\R) -- (-0.27*\L,0.6*\R) node[midway,above=3,above left=-1,scale=0.8] {4};
  
  % WIRE FRONT
  \foreach \i [evaluate={\x=-\L/2+(\i-1)*\w; \ang=atan(4*\R/(\w));}] in {1,...,\N}{
    \draw[metal] (\x,\R)++(-90-\ang:\t/2) --++ (90-\ang:\t) to[out=-\ang-5,in=185-\ang]++ (-\ang:2.02*\R)
                                          -- ({\x+\w/2},-\R) --++ (-90-\ang:\t/2) to[out=185-\ang,in=-\ang-5] cycle;
    \pic[scale=0.81] at ({\x+\w/2},-\R) {Bin={Icol}};
    \pic[scale=0.81] at (\x,\R) {Bout={Icol}};
  }
  \node[Icol] at (-0.545*\L,1.02*\R) {$I$};
%  \foreach \i [evaluate={\x=-\L/2+(\i-1)*\L/(\N-1); \ang=atan(2*\R*(\N-1)/\L)}] in {1,...,\N}{
%    \ifthenelse{\i<\N}{
%      \draw[lightmetal]
%        (\x,-\R)++(\ang+90:\t/2) --++ (\ang:2.08*\R) -- ({\x+\L/(\N-1)},\R) --++(\ang-90:\t/2) --++ (\ang-180:2.08*\R) ;
%    }{}
%    \draw[metal] (\x-\t/2,\R) arc (180:0:\t/2) --++ (0,-2*\R) arc (0:-180:\t/2) -- cycle;
%    \pic[scale=0.81] at (\x,-\R) {Bin={Icol}};
%    \pic[scale=0.81] at (\x,\R) {Bout={Icol}};
%  }
  
  \message{Loop with Ampere's law done. ^^J}
    
\end{tikzpicture}


% SOLENOID MAGNETIC FIELD
\def\xmax{6}
\def\ymax{3.5}
\begin{tikzpicture}[shift={(\xmax+0.024,\ymax+0.024)}]
  \message{Solenoid start. ^^J}
  \def\R{1}
  \def\N{8}
  \def\L{6.5}
  \def\t{0.11*\R}
  \begin{scope} %[shift={(3,3)}]
    \clip (-\xmax,-\ymax) rectangle (\xmax,\ymax);
    \begin{pspicture*}(-\xmax,-\ymax)(\xmax,\ymax) %dotangle=-90
      \rotatebox{-90}{
      \psframe[linecolor=white](-\ymax,-\xmax)(\ymax,\xmax)
      \psmagneticfield[
          N=\N,R=\R,L=\L,
          nL=4,pointsB=1000,%PasB=0.002,
          nS=1,numSpires=,PasS=0.001,pointsS=1200, %2 3 4 5 6 7,PasS=0.004
          linewidth=1.0pt,linecolor=Bcol,drawSelf=false
        ](-\ymax,-\xmax)(\ymax,\xmax)}
    \end{pspicture*}
  \end{scope}
  \foreach \i [evaluate={\x=-\L/2+(\i-1)*\L/(\N-1);}] in {1,...,\N}{
    \draw[metal] (\x-\t/2,\R) arc (180:0:\t/2) --++ (0,-2*\R) arc (0:-180:\t/2) -- cycle;
  }
  \message{Solenoid done. ^^J}
\end{tikzpicture}


\end{document}
